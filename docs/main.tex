\documentclass[12pt,a4paper]{article}
\usepackage{listings}
\usepackage{geometry}
\usepackage{setspace}
\usepackage{fancyhdr}
\usepackage{xcolor}
\usepackage{polyglossia}

\setdefaultlanguage{greek}
\setotherlanguage{english}
\setmainfont{Times New Roman}
\newfontfamily\greekfont{Times New Roman}
\newfontfamily\englishfont{Times New Roman}

\setlength{\headheight}{15pt}

\geometry{margin=2.5cm}
\doublespacing
\pagestyle{fancy}
\fancyhf{}
\fancyhead[R]{\thepage}
\fancyhead[L]{Έξυπνο Συμβόλαιο Διαχείρισης Στοιχείων Κατοικιδίων}


\begin{document}

% Define colors
\definecolor{solidityKeyword}{rgb}{0.0,0.2,0.6}   % dark blue
\definecolor{solidityType}{rgb}{0.6,0.1,0.6}      % purple
\definecolor{solidityComment}{rgb}{0.25,0.5,0.35} % green
\definecolor{solidityString}{rgb}{0.6,0.2,0.2}    % reddish

% Fix for monospace fonts in Greek/English
\newfontfamily\greekfonttt{Courier New}
\newfontfamily\englishfonttt{Courier New}

% Define Solidity language
\lstdefinelanguage{Solidity}{
  keywords={
    contract,library,interface,struct,event,enum,function,modifier,
    constructor,fallback,receive,returns,import,pragma,storage,memory,calldata,
    if,else,for,while,do,break,continue,return,emit,revert,require,assert,new,
    mapping,this,super,delete,using,as,assembly
  },
  keywordstyle=\color{solidityKeyword}\bfseries,
  ndkeywords={address,bool,string,bytes,int,uint,int8,uint8,int16,uint16,
    int32,uint32,int64,uint64,int128,uint128,int256,uint256,byte,ufixed,fixed},
  ndkeywordstyle=\color{solidityType}\bfseries,
  sensitive=true,
  comment=[l]{//},
  commentstyle=\color{solidityComment}\ttfamily,
  morecomment=[s]{/*}{*/},
  stringstyle=\color{solidityString}\ttfamily,
  morestring=[b]",
  morestring=[b]'
}

% Default style
\lstset{
  language=Solidity,
  basicstyle=\ttfamily\footnotesize,
  numbers=left,
  numberstyle=\tiny\color{gray},
  stepnumber=1,
  numbersep=5pt,
  showstringspaces=false,
  tabsize=2,
  breaklines=true,
  frame=single,
  rulecolor=\color{black!30},
  captionpos=b
}


\begin{titlepage}
    \centering
    \vspace*{2cm}
    
    {\huge\textbf{Έξυπνο Συμβόλαιο Διαχείρισης Στοιχείων Κατοικιδίων}}\\
    \vspace{2cm}
    {\large Προγραμματιστική Εργασία Solidity (2025)}
    \vspace{3cm}
    
    {\large\ \textbf{Μάθημα:} Τεχνολογίες Blockchain \& Εφαρμογές}\\
    {\large\ \textbf{Καθηγήτρια:} Δρ. Αριστέα Κοντογιάννη}\\
    \vspace{2cm}
    
    {\large Γιώργος Νικολαΐδης (π21115)}\\
\end{titlepage}

% - Η εργασια εμπεριέχει ένα αρχείο κώδικα Solidity 
% - Στο κώδικα ορίζεται ένα εξυπνο συμβόλαιο "Pets"
% - Το "Pets" contract καθορίζει 5 μεθόδους, μια για κάθε λειτουργία της εκφώνησης
\section{Εισαγωγή}
Η παρούσα εργασία εμπεριέχει ένα αρχείο κώδικο Solidity που ορίζεται ένα έξυπνο συμβόλαιο ονόματι "\textbf{Pets}". Το συμβόλαιο καθορίζει πέντε μεθόδους (functions), η κάθε μέθοδος υλοποιεί μια λειτουργία που ορίζει η εκφώνηση. 

Στις δύο ενότητες που ακολουθούν, θα αναλυθεί περαιτέρω ο κώδικας και θα φανεί σε πράξη η λειτουργία του έξυπνου συμβόλαιου.

\newpage

\section{Ανάλυση Κώδικα}

\subsection{Ο Κώδικας}
\lstinputlisting[language=Solidity]{../contracts/Pets.sol}

\subsection{Τα Μέρη του Συμβολαίου}
Το συμβόλαιο εσωτερικά είναι χωρισμένο σε διάφορα μέρη.

\subsubsection{Ορισμός Δομών Δεδομένων}
Στο συμβόλαιο γίνεται η χρήση ειδικών δομών δεδομένων (struct) για να επιτευχθεί η ορθή και ξεκάθαρη μοντελοποίηση του προβλήματος.

\begin{itemize}
    \item \textbf{Pet}: Μοντελοποιεί τις ιδιότητες ενός κατοικιδίου, όπως ορίζεται στην εκφώνηση της εργασίας. Δεν έχει πεδίο για τον μοναδικό κωδικό chip (Chip ID), η ταυτοποίηση με τον κωδικό γίνεται σε σημείο που θα δούμε αμέσως μέτα. Αξίζει να σημειωθεί η παρουσία των ιδιοτήτων \textbf{createdAt} και \textbf{modifiedAt}, η οποίες υπάρχουν για λόγους auditing - να μπορεί οποιοσδήποτε να δεί πότε δημιουργήθηκε/ενημερώθηκε ένα κατοικίδιο.
    
    \item \textbf{PetStatus}: Παρέχει πληροφορίες για την κατάσταση του κατοικιδίου - αν είναι ενεργό στο σύστημα και μια αιτιόλογηση για αν δεν είναι.
\end{itemize}

\subsubsection{Μεταβλητές Κατάστασης (State Variables)}


\end{document}